%% -*- ispell-local-dictionary: "italiano" -*-
%% Local Variables:
%% ispell-local-dictionary: "italiano"
%% eval: (flyspell-mode 1)
%% End:


\documentclass[a4paper, 12pt]{article}

\usepackage[utf8]{inputenc}
\usepackage[T1]{fontenc}
\usepackage{lipsum}
\usepackage{parskip}
\usepackage[a4paper,width=150mm,top=25mm,bottom=25mm]{geometry}

\usepackage{amsmath}
\usepackage{graphicx}
\usepackage{float}
\usepackage{subcaption}
\captionsetup{width=0.8\textwidth}

\usepackage{listings}
\usepackage{xcolor}
\lstset{
  basicstyle=\ttfamily,
  columns=fullflexible,
  breaklines=true,
  postbreak=\raisebox{0ex}[0ex][0ex]{\color{red}$\hookrightarrow$\space},
  keepspaces,
  language=Bash,
  showtabs=true,
}

\usepackage[noframe]{showframe}
\usepackage{framed}
\renewenvironment{shaded}{%
  \def\FrameCommand{\fboxsep=\FrameSep \colorbox{shadecolor}}%
  \MakeFramed{\advance\hsize-\width \FrameRestore\FrameRestore}}%
 {\endMakeFramed}
 \definecolor{shadecolor}{gray}{0.90}

\usepackage{hyperref}
\hypersetup{
  colorlinks=false,
  hidelinks=true,
  pdftitle={Compressione di immagini tramite la DCT}
}

\usepackage[T1]{fontenc}
\usepackage{titlesec,  color}
\usepackage{fix-cm}
\makeatletter
\newcommand\HUGE{\@setfontsize\Huge{50}{60}}
\makeatother
\titleformat{\chapter}
	{\scshape\LARGE\bfseries\HUGE}
	{\makebox[6pc][l]{\HUGE\thechapter\hfil\rule[-4pt]{0.5pt}{2pc}}}
	{0pt}
	{\LARGE}
	\titlespacing*{\chapter}{0pt}{0pt}{24pt}


\title{\textsc{\textbf{Compressione di immagini tramite la DCT}}}

\author{
		\begin{tabular}{cc}
				Lorenzo Olearo & Alessandro Riva \\
				\href{mailto:l.olearo@campus.unimib.it}{\texttt{\small{l.olearo@campus.unimib.it}}} &
				\href{mailto:a.riva86@campus.unimib.it}{\texttt{\small{\quad a.riva86@campus.unimib.it}}}
		\end{tabular}
}



\date{A.A. 2022-2023}


\begin{document}
\maketitle


\textit{Lo scopo di questo progetto è di utilizzare l'implementazione della trasformata
	DCT2 in un ambiente open source e di studiare gli effetti di un algoritmo di
	compressione di tipo JPEG (senza utilizzare una matrice di quantizzazione) sulle
	immagini in toni di grigio. Comprende l'implementazione di un codice e la
	scrittura di una relazione da consegnare al docente.}

\vspace{12pt}

\begin{shaded}
Tutto il codice sorgente per la realizzazione del progetto comprendente
consegne, test, esperimenti, grafici, dati e relazione è stato pubblicato al
seguente repository GitHub:
\url{https://github.com/LorenzoOlearo/cosine-image-compression}. Il repository
rimarrà privato fino alla data dell'esame, per avere l'accesso, le chiediamo per
ragioni tecniche di comunicarcelo via mail.
\end{shaded}

\renewcommand{\contentsname}{Indice dei contenuti}
\tableofcontents


\section{Confronto tra DCT diretta e veloce}
Nella prima parte del progetto si richiede di confrontare le prestazioni della
DCT2 come spiegata a lezione nella sua forma diretta, con quella di una libreria
open source a scelta che si presuppone essere nella sua versione \textit{fast}.

La trasformata DCT2 diretta è stata implementata in Python mentre per la sua
versione \textit{fast} è stata utilizzata la libreria Python \texttt{fftpack} di
\texttt{scipy}.

Per poter implementare la DCT2 è stato prima necessario implementare la
trasformata DCT. In questo progetto, è stata implementata la DCT di tipo II con
riferimento alla seguente definizione:

\begin{equation*}
	c_k = \alpha_k^N \sum_{i=0}^{N-1} x_i \cos \left(\pi k \frac{2i + 1}{2N} \right), \qquad k=0, \cdots, N-1
\end{equation*}

Dove, $\alpha_k^N$ è il fattore di normalizzazione ed è definito come:
\begin{equation*}
	\begin{cases}
		\alpha_k^N = \sqrt{\frac{1}{N}} \quad per \quad k = 0 \\
		\alpha_k^N = \sqrt{\frac{2}{N}} \quad per \quad k = 1, \cdots, N-1
	\end{cases}
\end{equation*}
in modo che le basi dello spazio dei coseni siano ortonormali.

L'implementazione Python della DCT come definita sopra è la seguente:
\begin{figure}[H]
	\centering
	\includegraphics[width=0.9\textwidth]{imgs/dct-python.png}
\end{figure}

Per calcolare la DCT2 su matrici si è calcolata la DCT, come definita sopra, prima
sulle righe e poi sulle colonne della matrice.
\begin{figure}[H]
	\centering
	\includegraphics[width=0.6\textwidth]{imgs/dct2-python.png}
\end{figure}


Per confrontare i tempi di esecuzione delle due implementazioni sono state
create una serie di matrici di interi di dimensioni crescenti, da 2x2 a
1024x1024, con valori casuali compresi tra 0 e 255. La scelta dell'upper-buond
dei test dipende soltanto da i limiti computazionali dei calcolatori a nostra
disposizione.

Ogni matrice è stata poi trasformata con entrambe le implementazioni della DCT2
tenendo traccia dei rispettivi tempi di esecuzione. I risultati sono stati
riportati su un grafico che mette in relazione le dimensioni delle matrici con i
tempi di esecuzione degli algoritmi in scala logaritmica.

\begin{figure}[H]
	\includegraphics[width=0.8\textwidth]{imgs/bench-incremental-0.png}
	\caption{Confronto tra DCT diretta e veloce, sull'asse delle ascisse la
		dimensione delle matrici, sull'asse delle ordinate il tempo di esecuzione in
		scala logaritmica.}
	\label{fig:incremental-benchmark}
\end{figure}

Come si può osservare dal grafico in Figura
\ref{fig:incremental-benchmark}, il tempo computazione della DCT calcolata
tramite il metodo diretto cresce in maniera decisamente più rapida rispetto a
quello della DCT calcolata tramite la libreria \texttt{fft} di \texttt{scipy}.

La DCT2 calcolata tramite il metodo diretto presenta tempi proporzionali a $N^3$
rendendola quindi inutilizzabile per matrici di grandi dimensioni, la DCT calcolata
tramite la libreria \texttt{fft} di \texttt{scipy} invece presenta tempi di esecuzione 
decisamente migliori.

Siccome la libreria \texttt{fft} di \texttt{scipy} utilizza la FFT, si è scelto
di mettere a confronto i tempi di esecuzione della trasformata DCT2 su input di
dimensioni pari a potenze di 2 crescenti.

\begin{figure}[H]
	\centering
	\includegraphics[width=0.8\textwidth]{imgs/bench-order-0.png}
	\caption{Confronto tra DCT diretta e veloce su input di dimensione pari a
		potenze di 2. Sull'asse delle ordinate il tempo in scala logaritmica,
		sull'asse delle ascisse le potenze di due delle dimensioni delle matrici.}
	\label{fig:order-benchmark}
\end{figure}

Anche in questo caso, come si può osservare dal grafico in figura
\ref{fig:order-benchmark}, la DCT implementata tramite il medoto diretto risulta
essere considerevolmente più lenta rispetto a quella implementata tramite la
libreria
\texttt{fft} di \texttt{scipy}.

Tramite il software sviluppato è possibile ripetere i test qui presentati. Per
effettuare il confronto tra la DCT diretta che è stata implementata e quella
\textit{fast} di \texttt{scipy} su matrici di dimensione incrementali comprese
tra un valore di lower e upper buond, si può eseguire il programma con i
seguenti argomenti:
\vspace{12pt}
\begin{lstlisting}[frame=single]
 > python main.py --nogui \
    --performance --incremental --lower <value> --upper <value>
\end{lstlisting}

Il software sviluppato permette inoltre di eseguire il confronto sulle
prestazioni delle due trasformate su matrici di dimensioni pari a multipli di 2
tramite i seguenti argomenti:
\vspace{12pt}
\begin{lstlisting}[frame=single]
 > python main.py --nogui \
    --performance --order --lower <value> --upper <value>
\end{lstlisting}

Siccome è stato necessario implementare la trasformata coseno diretta, al fine
di poterne verificare la correttezza è stato necessario introdurre dei metodi di
test appositi. Si è scelto di esporre questi metodi tramite il parametro
\texttt{test}.

E' possibile quindi eseguire solo il test della corretta implementazione della
trasformata coseno con i seguenti parametri:
\vspace{12pt}
\begin{lstlisting}[frame=single]
 > python main.py --nogui --test
\end{lstlisting}

Per avere tutti i parametri con le loro descrizioni è possibile usare il
parametro \texttt{help} come segue:
\vspace{12pt}
\begin{lstlisting}[frame=single]
 > python main.py --help
    Cosine Image Compression,
    Written By Lorenzo Olearo and Alessandro Riva, 2023

    options:
      -h, --help            Show this help message and exit
      --nogui               Run program with no GUI
      --test                Run test
      --performance         Run performance test
      --slow                Use direct cosine transform [...]
      --order               Test DCT on matrix on order of 2 [...]
      --incremental         Test DCT on matrix on incremental [...]
      --lower LOWER         Lower bound for performance tests
      --upper UPPER         Upper bound for performance tests
      --iterations          Iteration of the same tests
      --load                Load pre-computed data in order to [...]
\end{lstlisting}




\section{Compressione di immagini}
La seconda parte della consegna richiede la realizzazione di un'interfaccia
grafica che permetta di caricare un'immagine in formato \texttt{BMP} in toni di
grigio e di applicare un algoritmo di compressione JPEG senza però utilizzare una
matrice di quantizzazione.



Anche per la realizzazione di questa seconda parte si è scelto di utilizzare
Python, in particolare, l'interfaccia grafica è stata realizzata tramite la
libreria \texttt{tkinter}.

Il software implementato permette all'utente di caricare un'immagine
\texttt{BMP} in scala di grigi dal proprio filesystem sfruttando il \textit{file
picker} di \texttt{tkinter}.

Una volta caricata l'immagine all'interno dell'applicazione, l'utente è in grado
di specificare:

\begin{itemize}
  \item un intero $F$ corrispondente alla dimensione in pixel dei
        \textit{macro-blocchi} su cui effettuare la DCT2.
  \item un intero $d$ compreso tra $0$ e $2F - 2$ rappresentante il valore di
        taglio delle frequenza
\end{itemize}


\subsection{Backend}

Una volta specificati questi parametri tramite l'interfaccia grafica, la matrice
corrispondente all'immagine \texttt{BMP} caricata dall'utente viene passata
all'algoritmo di compressione.

L'algoritmo di compressione viene invocato dal modulo controller tramite il metodo
\texttt{bitmap\_to\_jpeg} 
\begin{figure}[h]
  \centering
  \includegraphics[width=1\textwidth]{./imgs/bitmap-to-jpeg.png}
  \label{fig:bitmap-to-jpeg}
\end{figure}

Per ogni macroblocco si applicano ora una serie di funzioni nel seguente ordine:
\begin{enumerate}
  \item \texttt{dct2}: applica la DCT2 al macroblocco, i pixel in eccesso
    vengono eliminati
  \item \texttt{frequency\_cut}: applica il taglio delle frequenze
  \item \texttt{idct2}: applica l'antitrasformata al macroblocco
  \item \texttt{clamp\_macro\_block}: arrotonda i valori all'intero più vicino,
    mettendo a zero quelli negativi e a 255 quelli superiori di 255
\end{enumerate}

Infine, tutti i macroblocchi vengono ricomposti nell'ordine corretto per formare
l'immagine compressa dal metodo \texttt{recompose\_macro\_blocks}.

\begin{shaded}
  \centering
  L'intero codice sorgete è pubblicato al repository GitHub: \\
  \url{https://github.com/LorenzoOlearo/cosine-image-compression}.

\end{shaded}


\subsection{Frontend}
L'interfaccia grafica presenta una barra di controllo mostrata in 
figura~\ref{fig:control-bar} dove sono presenti due pulsanti, uno per 
selezionare l'immagine da utilizzare e uno per avviare il processo di 
compressione con DCT. Dopo il caricamento dell'immagine il pulsante per avviare 
la compressione viene abilitato insieme ai campi per l'input dei parametri 
$F$ e $d$.

Una volta trasformata l'immagine, l'applicazione mostra fianco a fianco
l'immagine originale e quella su cui è stato eseguito l'algoritmo di
compressione delle frequenze. L'applicazione permette inoltre \texttt{zoom} e
\texttt{pan} delle due immagini simultaneamente.

\begin{figure}[h]
  \includegraphics[width=\textwidth]{./imgs/control-bar.png}
  \caption{Barra di controllo dell'applicazione.}
  \label{fig:control-bar}
\end{figure}

L'inserimento del parametro $F$, è limitato ai valori tra 1 e la dimensione 
dell'immagine, e lo slider per il parametro $d$, è limitato tra $0$ e $2F-2$.
Alla pressione del pulsante DCT viene eseguita la trasformata e viene mostrato
il risultato della compressione affiancato all'immagine originale come mostrato
in figura~\ref{fig:after-transform}.

\begin{figure}[h]
  \includegraphics[width=\textwidth]{./imgs/after-transform.png}
  \caption{L'applicazione mostra l'immagine originale e quella trasformata.}
  \label{fig:after-transform}
\end{figure}

Per meglio visualizzare l'immagine è possibile utilizzare la rotella del mouse 
per scalarla e il tasto sinisto per trascinarla. Così facendo è possibile
apprezzare la differenza tra le due immagini come in 
figura~\ref{fig:zoomed-detail}.

\begin{figure}[h]
  \includegraphics[width=\textwidth]{./imgs/zoomed-detail.png}
  \caption{Zoom su un dettaglio dell'immagine.}
  \label{fig:zoomed-detail}
\end{figure}

Lo zoom e il pan sono implementati tramite il modulo \texttt{PIL} di Python che
viene utilizzato per tagliare e ridimensionare le immagini prima di mostrarle, 
le trasformazioni utilizzano un criterio nearest neighbour, in modo da non 
introdurre artefatti durante lo zoom.


\begin{figure}[h]
  \includegraphics[width=\textwidth]{./imgs/deer-1.png}
  \caption{DCT con valore di taglio delle frequenze ad 1, in questo modo solo
    il valore medio non viene eliminato, dal confronto infatti, si può notare
    come in ogni macroblocco sia la media dell'intensità dei pixel che lo
    compongono.}
  \label{fig:deer-1}
\end{figure}


\begin{figure}[h]
  \includegraphics[width=\textwidth]{./imgs/deer-full.png}
  \caption{DCT con il valore massimo di taglio delle frequenze, si può notare come in questo caso non venga persa informazione in frequenza nell'immagine.}
  \label{fig:deer-full}
\end{figure}


\end{document}
